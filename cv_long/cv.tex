%%%%%%%%%%%%%%%%%%%%%%%%%%%%%%%%%%%%%%%%%
% Classicthesis-Styled CV
% LaTeX Template
% Version 1.0 (22/2/13)
%
% This template has been downloaded from:
% http://www.LaTeXTemplates.com
%
% Original author:
% Alessandro Plasmati
%
% License:
% CC BY-NC-SA 3.0 (http://creativecommons.org/licenses/by-nc-sa/3.0/)
%
%%%%%%%%%%%%%%%%%%%%%%%%%%%%%%%%%%%%%%%%%

%----------------------------------------------------------------------------------------
%	PACKAGES AND OTHER DOCUMENT CONFIGURATIONS
%----------------------------------------------------------------------------------------

\documentclass{scrartcl}

\usepackage{graphicx}

\reversemarginpar % Move the margin to the left of the page 

\newcommand{\MarginText}[1]{\marginpar{\raggedleft\itshape\small#1}} % New command defining the margin text style

\usepackage[nochapters]{classicthesis} % Use the classicthesis style for the style of the document
\usepackage[LabelsAligned]{currvita} % Use the currvita style for the layout of the document

\renewcommand{\cvheadingfont}{\LARGE\color{Maroon}} % Font color of your name at the top

\usepackage{hyperref} % Required for adding links	and customizing them
\hypersetup{colorlinks, breaklinks, urlcolor=Maroon, linkcolor=Maroon} % Set link colors

\newlength{\datebox}\settowidth{\datebox}{Spring 2011} % Set the width of the date box in each block

\newcommand{\NewEntry}[3]{\noindent\hangindent=2em\hangafter=0 \parbox{\datebox}{\small \textit{#1}}\hspace{1.5em} #2 #3 % Define a command for each new block - change spacing and font sizes here: #1 is the left margin, #2 is the italic date field and #3 is the position/employer/location field
\vspace{0.5em}} % Add some white space after each new entry

\newcommand{\Description}[1]{\hangindent=2em\hangafter=0\noindent\raggedright\footnotesize{#1}\par\normalsize\vspace{1em}} % Define a command for descriptions of each entry - change spacing and font sizes here

%----------------------------------------------------------------------------------------

\begin{document}

\thispagestyle{empty} % Stop the page count at the bottom of the first page




%----------------------------------------------------------------------------------------
%	major scientific achievements
%----------------------------------------------------------------------------------------

%----------------------------------------------------------------------------------------
%	NAME AND CONTACT INFORMATION SECTION
%----------------------------------------------------------------------------------------

\begin{cv}{\spacedallcaps{Babak Maboudi Afkham}}\vspace{1.5em} % Your name

\noindent\spacedlowsmallcaps{Personal Information}\vspace{0.5em} % Personal information heading

\NewEntry{}{\textit{Born on}}{22 March 1989} % Birthplace and date

\NewEntry{Nationality}{}{Iranian}

\NewEntry{email}{\href{mailto:babak.maboudi@epfl.ch}{babak.maboudi@epfl.ch}} % Email address

%\NewEntry{website}{\href{http://www.johnsmith.com}{http://www.johnsmith.com}} % Personal website

\NewEntry{phone}{(M) +41 78 627 46 97} % Phone number(s)

\vspace{1em} % Extra white space between the personal information section and goal

%\noindent\spacedlowsmallcaps{Goal}\vspace{1em} % Goal heading, could be used for a quotation or short profile instead

%\Description{Gain fundamental experience in my area of interest and expertise.}\vspace{2em} % Goal text

%----------------------------------------------------------------------------------------
%	Interests
%----------------------------------------------------------------------------------------

\spacedlowsmallcaps{Interests}\vspace{1em}

\Description{\MarginText{Research}Fast Numerical Solutions for Parametric Partial Differential Equations, via Model Order Reduction. Developing Structure-Preserving Model-Reduction Techniques for Hyperbolic Problems.}

\Description{\MarginText{Applied Mathematics}Model Order Reduction, Approximation Theory, Uncertainty Quantification, Inverse Problems, Machine Learning.}

\Description{\MarginText{Pure Mathematics}Differential Geometry, Symplectic Geometry, Statistics.}

\Description{\MarginText{Computer Science}Distributed and Parallel systems.}


%----------------------------------------------------------------------------------------
%	EDUCATION
%----------------------------------------------------------------------------------------

\spacedlowsmallcaps{Education}\vspace{1em}

\NewEntry{2014-present}{Ecole Polytechnique F\'ed\'erale de Lausanne (EPFL), Lausanne-Switzerland}

\Description{\MarginText{ \centering \includegraphics[height=0.075\textwidth]{./logos/epfl.png} }\textbf{Ph.D. in Computational Mathematics and Simulation Science} \newline
Advisor: Prof. Jan S. Hesthaven \newline
Research topic: Structure-Preserving Model-Reduction}

\Description{Abstract: Over the past decade, model order reduction has been successful in reducing the computational complexity of large scale and parametric elliptic and parabolic partial differential equations. Model order reduction of hyperbolic equation, however remains as a challenge. These problems are often derived from physical or geometrical symmetries, conservation laws and invariants. Such structures are lost over the course of conventional model reduction routines, e.g. proper orthogonal decomposition (POD), which often leads to unstable and qualitatively wrong solutions. In my thesis, we investigate an alternative model reduction approach to preserve the sympletic symmetry, an intrinsic structures of Hamiltonian systems. Unlike POD that constructs an orthonormal basis for the reduction space, this method constructs am orthogonal basis with respect to the pseudo inner product associated with Hamiltonian systems, known as the symplectic form. We show that the symplectic orthogonality provide a similar convergence rate as conventional orthogonal bases. Furthermore, we generalize the method to adopt to the norms and inner products most appropriate to the problem set up and discretizatoin. This allows the method to be applied on a wider range of discretization methods, especially the finite element methods. Moreover, we extend the method to consider dissipative systems: systems that are under the influence of dissipative forces but are Hamiltonian otherwise in the absence of such forces. 

}

\NewEntry{2017-2018}{Massachusetts Institute of Technology (MIT), Cambridge-United States of America}

\Description{\MarginText{\centering \includegraphics[height=0.06\textwidth]{./logos/mit.png}}\textbf{Exchange Graduate Student in Aeronautics and Astronautics} \newline
Advisor: Prof. Karen Willcox \newline
Research topic: Energy-Preserving Model-Reduction for Euler's Equation}

\NewEntry{2012-2014}{Royal Institute of Technology (KTH), Stockholm-Sweden}

\Description{\MarginText{\centering \includegraphics[height=0.1\textwidth]{./logos/kth.png}}\textbf{M.Sc. in Scientific Computing} \newline
Advisor: Prof. Anna-Karin Tornberg \newline
Thesis topic: Simulation of elastic rods with intrinsic curvature and twist immersed in fluid}

\NewEntry{2007-2012}{Sharif University of Technology (SUT), Tehran-Iran}

\Description{\MarginText{\centering \includegraphics[height=0.1\textwidth]{./logos/sut.png}}\textbf{B.Sc. in Theoretical Mathematics} \newline
Advisor: Prof. Mohammad Reza Razvan \newline
Thesis topic: Learning Spectral Clustering}

%------------------------------------------------


%----------------------------------------------------------------------------------------
%	Awards
%----------------------------------------------------------------------------------------

\spacedlowsmallcaps{Awards}\vspace{1em}

\Description{\MarginText{2017}The SNSF Doc.Mobility grant, 2017.}

\Description{\MarginText{2014}The SMC (Stockholm Mathematics Center) award for excellent master thesis, 2014.}

\Description{\MarginText{2013}KTH scholarship and tuition fee waiver, 2013.}

\newpage

%----------------------------------------------------------------------------------------
%	Publications
%----------------------------------------------------------------------------------------

\spacedlowsmallcaps{Publications}\vspace{1em}

\Description{\MarginText{2018}Babak Maboudi Afkham, Jan S. Hesthaven, "Structure-Preserving Model-Reduction of Dissipative Hamiltonian System", Journal of Scientific Computing (2018): 1-19}

\Description{\MarginText{2017}Babak Maboudi Afkham, Jan S. Hesthaven, "Structure-Preserving Model-Reduction of Parametric Hamiltonian System", SIAM Journal on Scientific Computing 39.6 (2017): A2616-A2644}

\vspace{1cm}

\Description{\MarginText{2018}Babak Maboudi Afkham, Ashish Bhatt, Bernard Haasdonk, Jan S. Hesthaven, "Symplectic Model Reduction with a Weighted Inner Product", Submitted to SIAM Journal on Scientific Computing}

\Description{\MarginText{2018}Babak Maboudi Afkham, Karen Willcox, Jan Hesthaven, "Energy Preserving Model Reduction of Fluid Flows" - Under Preparation }



%------------------------------------------------



%----------------------------------------------------------------------------------------
%	TEACHING & Supervision
%----------------------------------------------------------------------------------------

\spacedlowsmallcaps{Teaching and Supervision}\vspace{1em}

\Description{\MarginText{2014-2017}Principal Teacher Assistant of Analysis I and II: Holding 8 hours of lecture, Holding Exercise classes,Designing weekly exercise sheets}

\Description{\MarginText{2017}Co-supervisor of the master thesis: "Energy preserving model reduction of fluid dynamics", Nicolo Ripamonti}

\Description{\MarginText{2015}Supervisor of the semester project: "Hamiltonian formulation for non-conservative systems", Bozorgmehr Aminian}



%----------------------------------------------------------------------------------------
%	TALKS AND SEMINARS
%----------------------------------------------------------------------------------------



\spacedlowsmallcaps{Invited Talks at International Conferences and Workshops} \vspace{1em}

\Description{\MarginText{2018}MoRePaS 2018 Conference - Nantes, France \newline Keynote: "Model Order Reduction While Preserving a First Integral"}

\Description{\MarginText{2016}MORCIP - Workshop on Model Order Reduction for Control \& Inverse Problems, EPFL \newline Invited Speaker: "Structure-Preserving Model Reduction of Hamiltonian Systems"}

\Description{\MarginText{2016}ALOP - Workshop on Reduced Order Models in Optimization, The University of Trier \newline Invited Speaker: "Structure-Preserving Model Reduction of Hamiltonian Systems"}


%----------------------------------------------------------------------------------------
%	SUMMER SCHOOL & WORKSHOPS
%----------------------------------------------------------------------------------------


\spacedlowsmallcaps{Schools and Workshops} \vspace{1em}

\Description{\MarginText{2016}Winter School on Uncertainty Quantification, University of Basel, Switzerland}

\Description{\MarginText{2015}Bayesian Methods for Inverse Problems, University of Warwick, Uk.}

\Description{\MarginText{2015}International School on Model Reduction for Dynamical Control Systems, Dubrovnik, Croatia.}

\Description{\MarginText{2013}PDC Summer School: Introduction to High-Performance Computing, KTH, Stockholm, Sweden.}



%----------------------------------------------------------------------------------------
%	LANGUAGES
%----------------------------------------------------------------------------------------

\spacedlowsmallcaps{Languages}\vspace{1em}

\Description{English (Professional working proficiency), Persian (Mother Tongue), French (Intermediate Proficiency)}

\newpage
%----------------------------------------------------------------------------------------
%	HOBBIES
%----------------------------------------------------------------------------------------

\spacedlowsmallcaps{Hobbies}\vspace{1em}

\Description{Rock-climbing, Mountaineering (Mount Kilimanjaro 5895m, Mount Damavand 5678m), Distance Running}


%----------------------------------------------------------------------------------------
%	References
%----------------------------------------------------------------------------------------

\spacedlowsmallcaps{References}\vspace{1em}

\Description{Prof. Jan S. Hesthaven \newline Ecole Polytechnique F\'ed\'erale de Lausanne (EPFL)}

\vspace{1em}

\Description{Prof. Bernard Haasdonk \newline University of Stuttgart}

\end{cv}

\end{document}
